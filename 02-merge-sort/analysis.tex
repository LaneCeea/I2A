\documentclass[12pt]{article}
\usepackage[a4paper, total={17.18cm, 24.62cm}]{geometry}
\usepackage[onehalfspacing]{setspace}
\usepackage{amssymb}
\usepackage{amstext}
\usepackage{amsmath}
\usepackage{mathtools}
\usepackage{listings}
\usepackage{xcolor}

% Define the custom pseudocode style
\lstdefinestyle{pseudocode}{
    basicstyle=\ttfamily\normalsize,
    keywordstyle=\bfseries\color{blue},
    numberstyle=\small\color{gray},
    % Define your pseudocode keywords - fixed
    language={},
    keywords={if, else, while, for, to, break, return, and, or, in, True},
    % Frame and appearance
    frame=single,
    frameround=tttt,
    framesep=5pt,
    numbers=left,
    numbersep=10pt,
    breaklines=true,
    breakatwhitespace=false,
    tabsize=4,
    showstringspaces=false,
    captionpos=b,
    % Background color (optional)
    backgroundcolor=\color{gray!5},
    % Line spacing
    lineskip=2pt
}

% Create the pseudocode environment - FIXED: use lstnewenvironment
\lstnewenvironment{pseudocode}[1][]
{%
    \lstset{style=pseudocode, #1}%
}
{}

\DeclarePairedDelimiter\ceil{\lceil}{\rceil}
\DeclarePairedDelimiter\floor{\lfloor}{\rfloor}

\begin{document}

\section*{Problem}

\noindent\textbf{Input:} A sequence of \(n\) numbers \(\langle a_0, a_1, \dots, a_{n-1} \rangle\).

\noindent\textbf{Output:} A reordering \(\langle a_0', a_1', \dots, a_{n-1}' \rangle\) of the input sequence such that \(a_0' \leq a_1' \leq \dots \leq a_{n-1}'\).

\section*{Algorithm}

\begin{pseudocode}
merge(A[], T[], l, m, r)
    i = l
    p = l
    q = m
    while True
        if p >= m
            T[i .. r) = A[q .. r)
            break
        else if q >= r
            T[i .. r) = A[p .. m)
            break
        
        if A[p] <= A[q]
            T[i] = A[p]
            p++
        else
            T[i] = A[q]
            q++
        i++
    A[l .. r) = T[l .. r)

merge-sort-r(A[], T[], l, r)
    if l + 1 >= r
        return
    m = (l + r) // 2
    merge-sort-r(A, T, l, m)
    merge-sort-r(A, T, m, r)
    merge(A, T, l, m, r)

merge-sort(A[0 .. n))
    T = array with size n
    merge-sort-r(A, T, 0, n)
\end{pseudocode}

\newpage

\section*{Notation}

For the following proof, we have some predefined notations.
\begin{itemize}
    \item For any array \(A\) and integers \(l\) and \(r\), \(A[l : r)\) represents the subarray of \(A\) containing the element \(A[l], A[l+1], \dots, A[r-2], A[r-1]\), excluding \(A[r]\) (hence the open interval on r). For the case where \(l \geq r\), \(A[l : r)\) is defined as an empty array.
    \item When we use the interval notation \(i \in [l, r)\), it implies that \(i, l\) and \(r\) are integers and \(l \leq i < r\), not the usual real interval definition.
\end{itemize}

\section*{Correctness}

\subsection*{Merge}

We first prove that given the two sorted subarray and one working array:
\begin{center}
    \(A[l : m)\), \(A[m : r)\) and \(T[l : r)\), where \(l < m < r\),
\end{center}
the procedure \verb|merge| correctly merges them into the sorted subarray \(A[l : r)\).

Define the loop invariant \(\mathcal{L}\) as follow:

\(\mathcal{L}(i)\): At the start of the iteration \(i\) of the while loop at line 5-19, \(T[l : i)\) contains \((i - l)\) smallest elements from \(A[l : r)\) in sorted order. Moreover, \(\min(A[p], A[q])\) is the smallest element from \(A[l : r)\) that has not been copied into \(T\). If one of \(p\) and \(q\) is \textit{out of bounds} (\(p \in [l, m)\) or \(q \in [m, r)\)), then we define \(\min(A[p], A[q])\) as the one still in bounds. (If \(p \geq m\), then \(\min(A[p], A[q])\) is \(A[q]\). If \(q \geq r\), then \(\min(A[p], A[q])\) is \(A[p]\).)

It is not possible for both of \(p\) and \(q\) to be out of bounds in an iteration. We will discuss this in the Maintenance section.

\begin{enumerate}
    \item \textbf{Initialization:} Prove that \(\mathcal{L}(l)\) is true.

    At the start of the iteration \(i = l\), we have \(p = l\) and \(q = m\).
    \begin{enumerate}
        \item \(T[l : l)\) is an empty array. It does contain \(l - l = 0\) smallest elements from \(A[l : r)\).
        \item \(p = l < m\) and \(q = m < r\), so \(p \in [l, m)\) and \(q \in [m, r)\)
        \item Given the fact that \(A[l : m)\) and \(A[m : r)\) are sorted, \(A[l]\) is the smallest in \(A[l : m)\) and \(A[m]\) is the smallest in \(A[m : r)\), so \(\min(A[l], A[m])\) is the smallest element from \(A[l : r)\).
    \end{enumerate}

    \item \textbf{Maintenance:} Prove that for every \(i \in [l, r)\), if \(\mathcal{L}(i)\) is true, then \(\mathcal{L}(i+1)\) is true.
    Assume that \(\mathcal{L}(i)\) is true. For the case where at least one of \(p\) and \(q\) is \textit{out of bounds}, it triggers the termination condition at the start of the loop. We will discuss this in the Termination section. For now, we have the following assumption:
    \begin{enumerate}
        \item \(T[l : i)\) contains \((i - l)\) smallest elements from \(A[l : r)\) in sorted order.
        \item \(p\) and \(q\) are not \textit{out of bounds}.
        \item \(\min(A[p], A[q])\) is the smallest element from \(A[l : r)\) that has not been copied into \(T\).
    \end{enumerate}

    Now we have to ensure that \(\mathcal{L}(i+1)\) is true. Let's assume \(\min(A[p], A[q]) = A[p]\). Then, at the end of this iteration, we have:

    \begin{enumerate}
        \item \(T[i]\) is the next smallest element to be copied, which is \(A[p]\) based on the assumption. This is established through line 12-17. The smaller one of them gets stored in \(T[i]\). Thus, \(T[l : i+1)\) contains \((i + 1 - l)\) smallest elements from \(A[l : r)\) in sorted order.

        \item In line 12-17, \(A[p]\) gets stored in \(T[i]\) and \(p\) is increased by one, which may goes out of bounds. Note that only one of \(p\) and \(q\) gets increased by one in an iteration, so it is not possible for both of them to be out of bounds. For the case where \(p = m\), the subarray \(A[l : m)\) has all been copied into \(T\), so the elements that has not been copied into \(T\) are from the sorted subarray \(A[q : r)\). Thus, \(A[q]\) is the smallest element from \(A[l : r)\) that has not been copied into \(T\).

        \item For the case where \(p\) stays bounded, we have to ensure that \(\min(A[p], A[q])\) is still the smallest element from \(A[l : r)\) that has not been copied into \(T\). This is also established by the fact that the elements that has not been copied into \(T\) are from the sorted subarrays \(A[p : m)\) and \(A[q : r)\).
    \end{enumerate}

    Similarly, if \(\min(A[p], A[q]) = A[q]\), we have the same conclusion that \(\mathcal{L}(i+1)\) is true.

    \item \textbf{Termination:} In each iteration, either \(p\) or \(q\) increments, and both are bounded above, so the loop must terminate. At the iteration \(i = t\) where termination occurs, we have either \(p = m\) or \(q = r\). By the initialization and the maintenance of \(\mathcal{L}\), we have \(\mathcal{L}(t)\). Assume that \(p = m\), we have \(t = m + (q - m) = q\), then 
    \begin{enumerate}
        \item \(T[l : t)\) contains \((t - l)\) smallest elements from \(A[l : r)\) in sorted order.
        \item \(A[q]\) is the smallest element from \(A[l : r)\) that has not been copied into \(T\).
    \end{enumerate}

    Since \(A[l : m)\) has all been copied into \(T\), and \(A[q]\) is the smallest element from \(A[l : r)\) that has not been copied into \(T\), we can copied the whole the sorted subarray \(A[q : r)\) at \(T[t : r)\). Then the termination occurs, we have \(A[l : r)\) merged and sorted into \(T[l : r)\).

    Similarly, for the case where \(q = r\), \(t = p + (r - m)\), we have the conclusion that by copying the remaining \(A[p : m)\) to \(T[t : r)\), we have \(A[l : r)\) merged and sorted into \(T[l : r)\).
\end{enumerate}

At line 20, we copy \(T\) back to \(A\). Therefore, the procedure \verb|merge| correctly merge \(A[l : m)\) and \(A[m : r)\) into the sorted subarray \(A[l : r)\).

\subsection*{Merge Sort}

We prove by strong induction that for all \(n \geq 0\), the following \(P(n)\) is true.

\(P(n)\): for every integers \(l\) and \(r\), if \(n = r - l\), then \verb|merge-sort-r(A, T, l, r)| correctly sorts the subarray \(A[l : r)\) given a working array \(T[l : r)\).

\begin{enumerate}
    \item \textbf{Base Case:} When \(n \leq 1\), the function immediately returns without modifying any content. Since a subarray with size 0 or 1 is already sorted, \(P(0)\) and \(P(1)\) are both true.

    \item \textbf{Inductive Hypothesis:} Assume that for some integer \(k \geq 1\), \(P(i)\) is true for all integers \(i\) where \(0 \leq i \leq k\).

    \item \textbf{Inductive Step:} Prove that \(P(k+1)\) is true based on the inductive hypothesis.

    For every \(l\) and \(r\) such that \(r - l = k + 1\), since \(k+1 \geq 2\), that is \(r - l \geq 2\), the procedure goes through line 25-28. In line 25, we have
    \[
        m = \floor*{\frac{l + r}{2}} = \floor*{l + \frac{r - l}{2}} = l + \floor*{\frac{r - l}{2}} = l + \floor*{\frac{k + 1}{2}}.
    \]
    In line 26-27, the two subarrays are sorted based on the inductive hypothesis:
    \[
        \begin{array}{l@{\quad\quad\quad\quad}l}
            \begin{aligned}
                & m - l = \floor*{\frac{k + 1}{2}} \\
                & \text{Upper bound: } \floor*{\frac{k + 1}{2}} < k + 1 \\
                & \text{Lower bound: } \floor*{\frac{k + 1}{2}} \geq 1 \\
                & \Rightarrow 1 \leq m - l < k + 1, \\
                & \Rightarrow P(m - l) \text{ is true.}
            \end{aligned}
            &
            \begin{aligned}
                & r - m = k + 1 - \floor*{\frac{k + 1}{2}} = \ceil*{\frac{k + 1}{2}} \\
                & \text{Upper bound: } \ceil*{\frac{k + 1}{2}} < \frac{k + 1}{2} + 1 \leq k + 1 \\
                & \text{Lower bound: } \ceil*{\frac{k+1}{2}} \geq \frac{k + 1}{2} \geq 1 \\
                & \Rightarrow 1 \leq r - m < k + 1, \\
                & \Rightarrow P(r - m) \text{ is true.}
            \end{aligned}
        \end{array}
    \]
    In line 28, the procedure \verb|merge| correctly merge and sorted \(A[l : r)\) given the working array \(T[l : r)\) since \(A[l : m)\) and \(A[m : r)\) are sorted, and \(l < m < r\).
    
    In addition, the recursion terminates because each recursive call operates on a strictly smaller subarray, and the base case handles subarrays of size \(\leq 1\). Thus, \(P(k + 1)\) is true.

    \item \textbf{Conclusion:} By the principle of strong induction, \(P(n)\) is true for all integers \(n \geq 0\).
\end{enumerate}

Therefore, in the procedure \verb|merge-sort|, by creating a working array \(T[0 : n)\), the array \(A[0 : n)\) is correctly sorted by \verb|merge-sort-r(A, T, 0, n)|, thus solving the sorting problem.

\end{document}