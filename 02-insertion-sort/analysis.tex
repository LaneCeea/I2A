\documentclass[12pt]{article}
\usepackage[a4paper, total={17.18cm, 24.62cm}]{geometry}
\usepackage[onehalfspacing]{setspace}
\usepackage{amssymb}
\usepackage{amstext}
\usepackage{amsmath}
\usepackage{mathtools}
\usepackage{listings}
\usepackage{xcolor}

% Define the custom pseudocode style
\lstdefinestyle{pseudocode}{
    basicstyle=\ttfamily\normalsize,
    keywordstyle=\bfseries\color{blue},
    commentstyle=\itshape\color{gray},
    stringstyle=\color{red},
    numberstyle=\small\color{gray},
    % Define your pseudocode keywords - fixed
    language={},
    keywords={if, else, while, for, to, return, and},
    % Additional keywords for data types
    keywordstyle=[2]\color{purple}\bfseries,
    % Frame and appearance
    frame=single,
    frameround=tttt,
    framesep=5pt,
    numbers=left,
    numbersep=10pt,
    breaklines=true,
    breakatwhitespace=false,
    tabsize=4,
    showstringspaces=false,
    captionpos=b,
    % Background color (optional)
    backgroundcolor=\color{gray!5},
    % Line spacing
    lineskip=2pt
}

% Create the pseudocode environment - FIXED: use lstnewenvironment
\lstnewenvironment{pseudocode}[1][]
{%
    \lstset{style=pseudocode, #1}%
}
{}

\begin{document}

\section*{Problem}

\noindent\textbf{Input:} A sequence of \(n\) numbers \(\langle a_0, a_1, \dots, a_{n-1} \rangle\).

\noindent\textbf{Output:} A reordering \(\langle a_0', a_1', \dots, a_{n-1}' \rangle\) of the input sequence such that \(a_0' \leq a_1' \leq \dots \leq a_{n-1}'\).

\section*{Algorithm}

\begin{pseudocode}
insertion-sort(A[0 .. n-1])
    for i = 1 to n - 1
        key = A[i]
        j = i
        while j > 0 and A[j-1] > key
            j = j - 1
        if i != j
            A[j+1 .. i] = A[j .. i-1]
            A[j] = key
\end{pseudocode}

\section*{Correctness}

Let's define a \textbf{\textit{loop invariant}} \(\mathcal{L}\) to help us prove the correctness of the algorithm:

\(\mathcal{L}(i)\): At the start of the iteration \(i\) of the for loop, the subarray \(A[0 : i-1]\) contains the same elements originally in the input, but is now sorted.

\begin{enumerate}
    \item \textbf{Initialization:} The loop invariant is true before the first iteration of the loop.

    Trivially, \(A[0 : 0]\) is just a single element, thus is sorted and the same as the original input. \(\mathcal{L}(1)\) is true.

    \item \textbf{Maintenance:} If the loop invariant is true before an iteration of the loop, then it remains true before the next iteration.

    Suppose that \(\mathcal{L}(i)\) is true, that is, at the start of the iteration \(i\), the subarray \(A[0 : i-1]\) contains the same elements originally in the input and is sorted:
    \[
        A[0] \leq A[1] \leq \dots \leq A[i-1].
    \]
    In the while loop, we find the correct place to insert \(A[i]\). Specifically, there are two possible termination on line 5.
    \begin{enumerate}
        \item Terminates on \(j = 0\). In this case, we have that all elements from the subarray \(A[0 : i - 1]\) is greater than \(key\):
        \[
            key < A[0] \leq A[1] \leq \dots \leq A[i-1]
        \]

        \item Terminates on \(A[j - 1] \leq key\). This means that we found \(j\) such that
        \[
            A[0] \leq A[1] \leq \dots \leq A[j-1] \leq key < A[j] \leq \dots \leq A[i-1].
        \]
    \end{enumerate}

    Now we consider two cases at line 7.
    \begin{enumerate}
        \item If \(i = j\), then we have
        \[
            A[0] \leq A[1] \leq \dots \leq A[i-1] \leq A[i].
        \]
        The subarray \(A[0 : i]\) is sorted and unmodified in this iteration.

        \item If \(i \neq j\), then we move all the element from the subarray \(A[j : i-1]\) by one position to the right, and assign the original value of \(A[i]\) to \(A[j]\) via \(key\). Now we have
        \[
            A[0] \leq A[1] \leq \dots \leq A[j-1] \leq A[j] < A[j+1] \leq \dots \leq A[i].
        \]
        The subarray \(A[0 : i]\) is sorted. And it still contains the original elements since
        \begin{itemize}
            \item \(A[0 : j-1]\) is untouched;
            \item the new \(A[j+1 : i]\) is one-to-one to the original \(A[j : i-1]\);
            \item the new \(A[j]\) is the original \(A[i]\).
        \end{itemize}
    \end{enumerate}

    In both cases, after the iteration completes, the subarray \(A[0 : i]\) is sorted and contains the same elements as the original input \(A[0 : i]\). Therefore, at the start of the next iteration \(i+1\), \(\mathcal{L}(i+1)\) holds.

    \item \textbf{Termination:} When the loop terminates, the invariant gives us a useful property that helps show that that algorithm is correct.

    After the last iteration of the loop \(i = n-1\) completes, the maintenance property ensures that \(\mathcal{L}(n)\) is true: the subarray \(A[0 : n-1]\) contains the same elements originally in the input, but is now sorted. This is exactly the desired output.
\end{enumerate}

Therefore, the algorithm is correct and thus solve the sorting problem.

\end{document}